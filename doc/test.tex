\documentclass[a4paper, 10pt]{article}[5.10.2011]
%% packages
\usepackage[left=2cm, text={17cm, 24cm}, top=3cm]{geometry} % rozmery str�nky
\usepackage[czech]{babel}
\usepackage[latin2]{inputenc}
\usepackage[IL2]{fontenc}
\usepackage{colortbl}
\usepackage{graphicx}
\newcommand{\czuv}[1]{\quotedblbase#1\textquotedblleft}
\definecolor{gray}{rgb}{0.4,0.4,0.4}
% =======================================================================
% bal��ek "hyperref" vytv��� klikac� odkazy v pdf, pokud tedy pou�ijeme pdflatex
% probl�m je, �e bal��ek hyperref mus� b�t uveden jako posledn�, tak�e nem��e
% b�t v �ablon�
\usepackage{color}
\usepackage[unicode,colorlinks,hyperindex,plainpages=false]{hyperref}
\definecolor{links}{rgb}{0.4,0.5,0}
\definecolor{anchors}{rgb}{1,0,0}
\def\AnchorColor{anchors}
\def\LinkColor{links}
\def\pdfBorderAttrs{/Border [0 0 0] } % bez okraj� kolem odkaz�
\pdfcompresslevel=9




\title{Paraleln� a distribuovan� algoritmy\,--\,dokumentace \\Pipeline merge sort}
\author{Marek Kidon}
\date{\today}
\begin{document}
\maketitle
\noindent Dokumentace k 1.projektu do p�edm�tu Paraleln� a distribuovan� algoritmy (PRL). Obsahuje popis zad�n�, rozbor a anal�zu algoritmu Pipeline merge sort. V z�v�ru dokumentu se nach�z� komunika�n� protokol mezi \czuv{procesory} (zp�sob zas�l�n� zpr�v). Pro vizualizaci je vyu�it sekven�n� diagram.
\section{Zad�n�}
Pomoc� knihovny Open MPI implementujte algoritmus \textbf{Pipeline merge sort}.
\begin{description}
\end{description}


\normalsize
\end{document}


